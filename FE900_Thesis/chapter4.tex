\chapter{FPGA System Architecture and Methodology}

To validate the hypothesis that an AI-integrated FPGA can outperform traditional market-making strategies, we propose a hybrid system architecture. This architecture is founded upon a deterministic, pure FPGA tick-to-trade (T2T) pipeline by replacing its static decision logic with a hardware-accelerated reinforcement learning (RL) inference core.

This design is composed of two primary components, both specified in the underlying technical architecture:
\begin{itemize}
    \item \textbf{The Ultra-Low-Latency (ULL) Data Path:} A "pure-in-gates" FPGA pipeline responsible for all operations on the critical path, from network ingress to order egress.
    \item \textbf{The Hybrid Control Plane:} A software-based system (running on a host CPU) responsible for training the RL model and updating its parameters on the FPGA via a non-critical control path.
\end{itemize}

\section{The Ultra-Low-Latency T2T Data Path (FPGA)}

Our data path foundation is a deterministic, end-to-end T2T pipeline architecture. This design ensures our system operates with deterministic, sub-microsecond latency, eliminating OS jitter and software overheads.

\subsection{Network Ingress & Parsing (rx_parser)}
The system ingests 10/25GbE market data feeds directly from the PHY. The \texttt{rx_parser} module implements a Finite State Machine (FSM) to strip headers and extract the payload.

\begin{figure}[h!]
    \centering
    \caption{State Machine for RX Parser}
    \begin{tikzpicture}[->,>=stealth',shorten >=1pt,auto,node distance=2.5cm, semithick]
      \tikzstyle{every state}=[fill=blue!10,draw=none,text=black]

      \node[state, initial] (ETH) {ST_ETH};
      \node[state] (IP) [right of=ETH] {ST_IP};
      \node[state] (UDP) [right of=IP] {ST_UDP};
      \node[state] (PAY) [right of=UDP] {ST_PAY};

      \path 
        (ETH) edge node {Cycle 0} (IP)
        (IP) edge node {Cycle 1} (UDP)
        (UDP) edge node {Cycle 2} (PAY)
        (PAY) edge [loop above] node {Payload Stream} (PAY)
              edge [bend left] node {tlast=1} (ETH);
    \end{tikzpicture}
\end{figure}

The FSM cycles through:
\begin{enumerate}
    \item \textbf{ST_ETH:} Consumes the 14-byte Ethernet header.
    \item \textbf{ST_IP:} Extracts Source/Dest IP from the 20-byte IPv4 header.
    \item \textbf{ST_UDP:} Extracts Source/Dest Ports from the 8-byte UDP header and asserts \texttt{header_valid}.
    \item \textbf{ST_PAY:} Streams the remaining payload data to the decoder until \texttt{tlast} is asserted.
\end{enumerate}

\subsection{Feed Handling & Book Building (itch_decoder)}
The \texttt{itch_decoder} module is responsible for identifying and parsing specific market data messages.

\begin{figure}[h!]
    \centering
    \caption{ITCH Decoder Logic Flow}
    \begin{tikzpicture}[node distance = 1.5cm, auto]
        \tikzstyle{block} = [rectangle, draw, fill=green!10, text width=8em, text centered, rounded corners, minimum height=3em]
        \tikzstyle{line} = [draw, -latex', thick]
        
        \node [block] (input) {Input Stream};
        \node [block, right=of input] (buffer) {Accumulation Buffer (512-bit)};
        \node [block, below=of buffer] (check) {Length Check (>288 bits)};
        \node [block, right=of check] (extract) {Field Extraction};
        \node [block, below=of extract] (output) {Normalized Tick};

        \path [line] (input) -- (buffer);
        \path [line] (buffer) -- (check);
        \path [line] (check) -- node {Yes} (extract);
        \path [line] (extract) -- (output);
    \end{tikzpicture}
\end{figure}

Logic flow:
\begin{itemize}
    \item \textbf{Accumulation:} Incoming 64-bit words are shifted into a large internal buffer (`buffer`).
    \item \textbf{Detection:} The logic checks if the buffer contains enough bits for a full "Add Order" message (Type 'A', 36 bytes).
    \item \textbf{Extraction:} Once valid, it extracts the Order ID, Price, and Quantity fields using fixed bit-offsets and generates a single-cycle \texttt{tick_valid} pulse.
\end{itemize}

\section{Implemented Core: The RL-Inference Module}

A major contribution of this thesis is the implementation of the \texttt{strat_decide.v} module, which replaces static threshold logic with a hardware-accelerated RL inference core. This module has been synthesized and integrated into the pipeline.

\begin{itemize}
    \item \textbf{Inputs:} The module receives high-speed BBO signals (\texttt{bid_px0}, \texttt{ask_px0}) alongside real-time state features: Inventory Position and Market Volatility.
    \item \textbf{Architecture:} The core implements a 4-stage pipelined neural network:
    \begin{enumerate}
        \item \textbf{Feature Extraction:} Calculates spread ($Ask - Bid$), Order Book Imbalance (OBI), and inventory skew.
        \item \textbf{DSP MAC:} Uses Xilinx DSP48 slices to perform parallel dot-product operations of the feature vector against weights stored in Block RAM (BRAM).
        \item \textbf{Activation:} Applies a hardware-optimized ReLU function (max(0, x)).
        \item \textbf{Decision:} A comparator stage thresholds the activation output to generate \texttt{buy} or \texttt{sell} signals.
    \end{enumerate}
    \item \textbf{Latency:} The pipelined design achieves an inference latency of \textbf{4 clock cycles} (approx. 13.3ns at 300MHz), negligible compared to the total T2T budget.
\end{itemize}

This design retains the safety of the original pipeline. The AI's decisions are strictly gated by the \texttt{risk_gate} module, which enforces hard limits (e.g., Max Notional) in hardware, providing a fail-safe against model hallucinations.

\section{The Hybrid Control Plane for Model Management}

To bridge the gap between static hardware and dynamic markets, we implemented a PCIe-based control plane.

\begin{enumerate}
    \item \textbf{Model Deployment:} The CPU-based software stack trains the RL model and writes optimized weights into the FPGA's AXI-Lite registers. This allows for dynamic model updates without recompiling the bitstream.
    \item \textbf{Telemetry and Monitoring:} The FPGA exports high-resolution timestamps and latency histograms via the same PCIe path, enabling precise T2T benchmarking.
\end{enumerate}

This hybrid approach ensures the critical trading path remains purely in hardware and is \textit{never} back-pressured by the software plane.